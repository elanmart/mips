Software prepared for this thesis is meant to be run on a modern GNU/Linux distribution.
Specifically, we tested it on Ubuntu but it should be possible to compile and run the library on other distributions as long as necessary libraries that we use can be installed.
Here we describe how to prepare clean and up-to-date Ubuntu 16.04 LTS installation to use our library.
Commands should be run from terminal emulator.
\subsection{Basic configuration}
\noindent
First, install necessary packages.
\begin{minted}{bash}
sudo apt install git libopenblas-dev python-numpy \
python-dev python3-numpy python3-dev swig
\end{minted}
Clone the repository from GitHub together with faiss or get it from the volume supplied to the thesis.
\begin{minted}{bash}
git clone --recursive https://github.com/walkowiak/mips
\end{minted}
Copy faiss' makefile and customize it.
\begin{minted}{bash}
cp mips/faiss/example_makefiles/makefile.inc.Linux mips/makefile.inc
nano mips/makefile.inc
\end{minted}
Main changes concern \texttt{BLASLDFLAGS} and \texttt{PYTHONCFLAGS}.
These variables should be adapted to your system but in general they are going to look like this (pick one version of \texttt{PYTHONCFLAGS}).
\begin{minted}{bash}
BLASLDFLAGS?=/usr/lib/libopenblas.so.0
PYTHONCFLAGS=-I/usr/include/python3.5/ \
 -I/usr/lib/python3/dist-packages/numpy/core/include
PYTHONCFLAGS=-I/home/<username>/anaconda3/include/python3.6m/ \
 -I/home/<username>/anaconda3/lib/python3.6/\
site-packages/numpy/core/include/
\end{minted}
Using Anaconda, as shown above, can be helpful (Python 3.6 is not officially available in Ubuntu 16.04 repositories). You can install it by downloading script from \url{https://www.anaconda.com/download/}.
Allow execution of the script and run the installer.
\begin{minted}{bash}
chmod u+x Anaconda3-5.0.1-Linux-x86_64.sh
./Anaconda3-5.0.1-Linux-x86_64.sh
\end{minted}
You will have to accept the license terms, choose location for Anaconda (e.g. \texttt{/home/ <username>/anaconda3}) and allow it to be added to \texttt{PATH}.\\
Now you can compile our library and faiss' Python wrapper.
\begin{minted}{bash}
cd mips
make -j10
cd faiss
make py
\end{minted}
\subsection{Benchmarking on sift dataset}
\noindent
Download sift dataset and extract it to predefined directory.
\begin{minted}{bash}
cd ..
wget ftp://ftp.irisa.fr/local/texmex/corpus/sift.tar.gz
tar xf sift.tar.gz -C data
rm sift.tar.gz
mv data/sift data/sift1M
\end{minted}
This dataset's ground truth is in nearest neighbour format.
We need to transform it to MIPS groundtruth.
\begin{minted}{bash}
export PYTHONPATH=faiss
python3 python/misc/make_gt_IP.py --skip-tests data/sift1M
\end{minted}
Benchmark our implementations of algorithms.
\begin{minted}{bash}
bin/bench_kmeans 2 1 0.85 0 40 80 120
bin/bench_quant 128 32 0 0 0
bin/bench_alsh 64 96 10 1 0.85
\end{minted}
\subsection{Benchmarking on LSHTC dataset}
\noindent
Install needed Python libraries using Anaconda.
\begin{minted}{bash}
conda install pytorch -c pytorch
conda install -c conda-forge tinydb pybind11
\end{minted}
Compile bindings and fork of fastText.
\begin{minted}{bash}
make bind
cd fastText
make -j10
\end{minted}
Download WikiLSHTC-325K dataset from \url{http://manikvarma.org/downloads/XC/XMLRepository.html}.
Extract the archive, rename files to \texttt{test.txt} and \texttt{train.txt} and move them to \texttt{mips/data/LSHTC} directory.
Run Jupyter Notebook and follow the instructions.
\begin{minted}{bash}
jupyter-notebook tests/ours-on-wiki-test.ipynb 
\end{minted}
