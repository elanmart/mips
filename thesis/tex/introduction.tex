\section{\remark{Introduction to data science and machine learning}}

Today’s world is data-rich. It is almost impossible to analyze so big chunks of information using traditional data processing software.
Machine learning comes as solution, giving an alternative to analyzing huge volumes of data.
This field of computer science became more and more popular because of large datasets, which are used almost everywhere.
Machine learning is a data science technique, which enable computers to learn like humans or animals. It means that machines use algorithms to learn from data how to find a particular information and they are not explictly programmed how to do that.
That type of algorithms improves its operation when the number of training samples increases.
With the rise in big data it is used in such areas: natural language processing, image processing, recommendation systems, computational biology. Machine learning is applied in instances that we use in ordinary life like online recommendation systems --- friends suggestions on Facebook, Netflix movies suggestions based on our earlier choices and also it is used to tag documents for example on Wikipedia website.
It should be consider to use for complex task involving a large amount of data, which don’t have an exact equation. [TODO - trochę nie rozumiem :)]
Machine learning uses two popular methods: supervised learning and unsupervised learning.
Supervised learning train model uses known input and output data so that it can predict outputs for future inputs.
It uses classification and regression methods to develop models.
Classification techniques classify input data into categories, they are used in medical imaging and speech recognition.
Regression techniques predict continuous responses, example usage can be predicting house prices. Unsupervised learning groups and interprets data based only on given input.
The most common unsupervised learning technique is clustering.
It analyses input data to find a pattern in it. Typical usage of clustering method are recommendation systems. In our thesis we try to resolve a problem called MIPS using supervised learning classification method.
